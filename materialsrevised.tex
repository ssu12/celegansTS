\listfiles

\documentclass[11pt]{article}
\usepackage[margin=2cm]{geometry}

\usepackage{bm}			% Allows access to bold math symbols within math expressions
\usepackage{booktabs}		% Provides additional commands to enhance the quality of table in LaTeX
\usepackage{pdfpages}		% Allows external PDF pages to be inserted
%\usepackage{hyperref}		% Allows for hypertext links this includes embedding of links to websites
\usepackage{titlesec}		% Allows basic changes to the standard title (``chapter'') style, such as font style, size or placement of title

%\usepackage{color}			% Allows the addition of color to text and background			
%\usepackage{soul}			% Allows the highlighting/emphasizing text (need to load color package)
%\usepackage{hyphenat}		% Allows for the disabling or enabling of hypenation

%\usepackage{subfigure}
%\usepackage{float}	
%\usepackage{graphicx}	
%\usepackage{epstopdf}
%\usepackage{caption}		% Allows many aspects of the caption to be modified


\usepackage{amsmath}	
\usepackage{amssymb}	

\setlength\parindent{0pt}

\usepackage{siunitx}
\DeclareSIUnit{\molar}{M}

\usepackage{fixltx2e}
\usepackage[section]{placeins}
\usepackage{mhchem}
\usepackage{enumitem}

%matlab2tikz
\usepackage{pgfplots}
\pgfplotsset{compat=newest}
\usetikzlibrary{plotmarks}
\usetikzlibrary{arrows.meta}
\usepgfplotslibrary{patchplots}
\usepackage{grffile}
\usepackage{amsmath}
\newlength\figH
\newlength\figW
\setlength{\figH}{3in}
\setlength{\figW}{6.15in}


%Section formatting modifications
\titlespacing*{\subsection}{0pt}{15pt}{12pt}
\titlespacing*{\subsubsection}{0pt}{5pt}{0pt}

\titleformat{\section}{\normalfont\fontsize{12}{5}\bfseries}{\thesection}{1em}{}
  
%Formatting modifications for figures
%\captionsetup[figure]{labelfont=bf}

%Formatting modifications for tables
%\captionsetup[table]{labelfont=bf}

%Formatting modifications for footnotes
%\renewcommand{\thefootnote}{\roman{footnote}}
%\renewcommand{\theenumii}{\alph{enumii}}


\title{\vspace{-1em}{\underline{Independent Laboratory: Materials List}}\vspace{-.5em}}
\author{Benjamin Huang \& Shiye Su}
\date{}

\begin{document}

\maketitle

\section{Glassware \& Non-Disposables}

\begin{itemize}

	\item 1 pair of tweezers: for picking up worms.

	\item pipettes
	
\end{itemize}


\section{Disposables}

\begin{itemize}
	
	\item pipette tips

\end{itemize}

\section{Chemicals}

\begin{itemize}
	
	\item NaCl solution: sufficient to create pillars at concentration \SI{200}{\milli \molar}.

	\item Methylene blue, for colouring the high concentration pillars.

	\item 60 agar plates, 20 for the first day
		
\end{itemize}


\section{Biologicals}

\begin{itemize}

	\item 60 \emph{C. elegans} worms, 20 for the first day: provided in the same way they were in the chemotaxis lab. We would like them grown in \SI{50}{\milli \molar} NaCl solution. 
	
\end{itemize}


\section{Instruments}

\begin{itemize}

	\item 2 cameras: image sequence acquisition. 

	\item 2 computers: image sequence acquisition
	
	\item 1 ruler: for scale conversion.
	
\end{itemize}


\section{Miscellaneous}

\begin{itemize}
		
	\item skewer or needles: creating holes in agar plates in the location of cities. (the metal sticks you showed us)
		
	\item 2 retort stands with clamp: for holding camera.
	
	\item 2 LED rings: for imaging \emph{C. elegans}. Our optics set up for this part of the lab should be similar to that in the chemotaxis lab.
		
\end{itemize}


\stepcounter{footnote}\footnotetext{We won't need the stuff we previously wanted for the diffusion: toothepicks, plastic slab, petri dishes, cardboard, brilliant blue dye, etc}

\stepcounter{footnote}\footnotetext{We have requested 2 of many items because we hope to work simultaneously on the same tasks.}

\end{document}